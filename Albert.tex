\documentclass{tufte-handout}
\usepackage{enumitem}
\title{Formalising techniques for improvising and storytelling for children's bedtime stories}
\author[Tom Wallis]{Tom Wallis}
\begin{document}
\maketitle

\begin{abstract}\marginnote{You can find out lots about Project Albert at its homepage: \href{http://projectalbert.net}{http://projectalbert.net}}
    Storytelling is a valuable, yet sadly difficult-to-master art. It can allow groups to bond, and can be particularly valuable for parents, who often do not have the money to buy children's books to read.
    However, its difficulty-to-master means storytelling is a rarely used technique. Project Albert provides a fix for this by demonstrating a framework, built on design patterns, which can assist a storyteller in constructing a coherant, consistant story without becoming overly complex.
\end{abstract}

\section{Introduction}
\newthought{Stories are important.} For example, they're a terrific way to bond --- particularly for children. Many of us can relate to the experience of being much smaller and telling stories by firelight, or of being read to, or of playing make-believe as and inventing our own tales as we went along. \par

Fortunately, there are lots of good stories for children, written and illustrated by brilliant creatives, who bring joy to many young people\sidenote{Some of these people, like \href{https://en.wikipedia.org/wiki/Jon_Klassen}{Jon Klassen}, also bring joy with similar stories to grown-ups.}. However, not everybody can afford to buy children books, or have the aptitude and training to improvise stories. In these situations, a treasured and valuable experience is lost. Of course, solutions to this problem exist: libraries, television and audiobooks, and so on. However, sometimes, nothing beats a well-told story.\par

\newthought{This is especially true} of improvised stories. \par
An improvised story has a ``special something'' about it --- because it's unique, and because unlike stories written by professionals elsewhere, improvised stories can contain specific elements of the life of the audience\sidenote{While it's true that relatable elements of well-written stories \emph{are} elements of the audiences' lives, these elements cannot be specific and tailored.}. This means that an improvised story can have a greater potential as an agent of bonding between the storyteller and the audience than the standard story.\par

Unfortunately, a storytelling is a difficult skill to develop --- particularly without some natural aptitude. If it were possible for \emph{anyone} to simply ``have'' that skill, then the barrier to entry would be significantly lower. Were this the case, the value of the professionally-written children's story would remain\sidenote{There are advantages to professionally written stories in that they can be illustrated, can be involved and complicated in a way which is difficult to improvise, and can be on specific topics which an improviser does not necessarily know about, for starters}, but meaningful stories with the power to bond would be available to everyone.\par

To bridge this divide between skill required and skill present, some system for guiding a storyteller through their task could, in theory, be developed. That system would need to:

\begin{enumerate}
    \item Show how plot can be constructed in a way which is easy to improvise, while being coherant for the audience.
    \item Show how characters should be constructed, in a way which works within this plot. These characters would have to be:
        \begin{enumerate}
            \item Believable for the audience
            \item Relatable for the audience
            \item Consistent within the narrative
            \item Easy to construct for the storyteller
        \end{enumerate}
    \item Show how the plot and characters can exist within a world which is defined on-the-fly. This world would have to be:
        \begin{enumerate}
            \item Easy to construct for the storyteller
            \item Simple enough to reuse for later stories
            \item Believable for the audience
            \item A sufficient degree of detail to permit the story to be told properly, without becoming too complex to keep easily consistent.
        \end{enumerate}
\end{enumerate}

\newthought{To this end}, \emph{Project Albert} attempts to bring the value and power of improvisation to the layperson. It achieves this by provifing a framework, using design patterns, which act as a flexible scaffolding for the storyteller to weave their story around. This scaffolding, in theory, provides a fix for the problems outlined here, and acts as a proof of concept of improvisation-by-design-pattern which can be investigated further and developed upon. This essay provides exposition of the key concepts surrounding Project Albert, explores its usage, and suggests future work which can be done to further the progress in this experimental storytelling technique.

\subsection{Terminology}
\marginnote{You can find out more about all of the terms defined here at the \href{http://projectalbert.net/terms/}{Project Albert Terms page}.}
To clarify some terminology:
\begin{description}[align=right,labelwidth=3cm]
    \item [Design Pattern:] A design pattern is a way of managing solutions to common problems. It's used in architecture and software engineering to discuss a solution to a problem that tends to be:
		\begin{itemize}
			\item Flexible enough to solve lots of problems that match some archetype
			\item Robust enough to be changed and still solve a problem
        \end{itemize}\end{description}


\section{Design Patterns and Experimental Storytelling}

\subsection{Experimental Storytelling}

\subsection{Design Patterns}

\subsection{Merging the two}

\section{Albert patterns}

\subsection{Patterns for Plot}

\subsection{Patterns for Character Development}

\subsection{Patterns for Worldbuilding}

\section{Conclusion}

\section{Future work}

\appendix
\section{Appendix}

\subsection{Example stories}

\end{document}