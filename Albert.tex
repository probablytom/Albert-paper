\documentclass{tufte-handout}
\title{Formalising techniques for improvising and storytelling for children's bedtime stories}
\begin{document}
\maketitle
\begin{abstract}
    Since time immemorial, stories have been an essential and unavoidable component of human culture. Lately, this part of human culture has been fundementally changed by the introduction of a new technology: written stories which can be retold and reused.
    While there is an obvious benefit to this, we now often lack an old skill: improvised storytelling. As a result of the importance of storytelling to human culture, however, this lacking skill is a great shame, and there is no obvious path to gaining it back. 
    Project Albert attempt to fix this rift in culture, by making storytelling easier through the following of simple rules regarding the story. Built on knowledge from architecture and software engineering, Albert simplifies the construction and delivery of an improvised story which is easy to follow for the audience and easy to construct for the storyteller. To do this, we narrow our scope and focus on improvising children's bedtime stories, as these are brief, simple, and often formulaic, making them an ideal candidate for a formalised technique.
\end{abstract}
\section{Introduction}

\section{Formalising techniques}

\section{Design patterns}

\section{Albert patterns}

\section{Conclusion}

\section{Future work}

\end{document}