\documentclass{tufte-handout}
\title{Formalising techniques for improvising and storytelling for children's bedtime stories}
\begin{document}
\maketitle
\begin{abstract}
    Since time immemorial, stories have been an essential and unavoidable component of human culture. Lately, this part of human culture has been fundementally changed by the introduction of a new technology: written stories which can be retold and reused.
    While there is an obvious benefit to this, we now often lack an old skill: improvised storytelling. As a result of the importance of storytelling to human culture, however, this lacking skill is a great shame, and there is no obvious path to gaining it back. 
    Project Albert attempt to fix this rift in culture, by making storytelling easier through the following of simple rules regarding the story. Built on knowledge from architecture and software engineering, Albert simplifies the construction and delivery of an improvised story which is easy to follow for the audience and easy to construct for the storyteller. To do this, we narrow our scope and focus on improvising children's bedtime stories, as these are brief, simple, and often formulaic, making them an ideal candidate for a formalised technique.
\end{abstract}
\section{Introduction}
\newthought{Stories are important.} For example, they're a terrific way to bond --- particularly for children. Many of us can relate to the experience of being much smaller and telling stories by firelight, or of being read to, or of playing make-believe as and inventing our own tales as we went along. \par

Fortunately, there are lots of good stories for children, written and illustrated by brilliant creatives, who bring joy to many young people\sidenote{Some of these people, like \href{https://en.wikipedia.org/wiki/Jon_Klassen}{Jon Klassen}, also bring joy with similar stories to grown-ups.}. However, not everybody can afford to buy children books, or have the aptitude and training to improvise stories. In these situations, a treasured and valuable experience is lost. Of course, solutions to this problem exist: libraries, television and audiobooks, and so on. However, sometimes, nothing beats a well-told story.\par

\newthought{This is especially true} of improvised stories. \par
An improvised story has a ``special something'' about it --- because it's unique, and because unlike stories written by professionals elsewhere, improvised stories can contain specific elements of the life of the audience\sidenote{While it's true that relatable elements of well-written stories \emph{are} elements of the audiences' lives, these elements cannot be specific and tailored.}. This means that an improvised story can have a greater potential as an agent of bonding between the storyteller and the audience than the standard story.\par

Unfortunately, a storytelling is a difficult skill to develop --- particularly without some natural aptitude. If it were possible for \emph{anyone} to simply ``have'' that skill, then the barrier to entry would be significantly lower. Were this the case, the value of the professionally-written children's story would remain\sidenote{There are advantages to professionally written stories in that they can be illustrated, can be involved and complicated in a way which is difficult to improvise, and can be on specific topics which an improviser does not necessarily know about, for starters}, but meaningful stories with the power to bond would be available to everyone.\par

To bridge this divide between skill required and skill present, some system for guiding a storyteller through their task could, in theory, be developed. That system would need to:

\begin{enumerate}
    \item Show how plot can be constructed in a way which is easy to improvise, while being coherant for the audience.
    \item Show 
\end{enumerate}

\newthought{To this end}, \emph{Project Albert} attempts to bring the value and power of improvisation to the layperson. It achieves this by provifing a framework, using design patterns, which act as a flexible scaffolding for the storyteller to weave their story around. This scaffolding, in theory, provides a fix for the problems outlined here, and acts as a proof of concept of improvisation-by-design-pattern which can be investigated further and developed upon. This essay provides exposition of the key concepts surrounding Project Albert, explores its usage, and suggests future work which can be done to further the progress in this experimental storytelling technique.

\section{Formalising techniques}

\section{Design patterns}

\section{Albert patterns}

\section{Conclusion}

\section{Future work}

\end{document}